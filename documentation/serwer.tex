\section{Konfiguracja serwera}
\begin{enumerate}
    \item Środowisko i konfiguracja:
        \begin{itemize}
            \item dotenv - Wykorzystanie pakietu dotenv do zarządzania 
            zmiennymi środowiskowymi, co ułatwia konfigurację aplikacji bez bezpośredniego umieszczania wrażliwych danych w kodzie.
            \item Port - Serwer jest skonfigurowany do nasłuchiwania na porcie określonym w zmiennej środowiskowej PORT, z domyślnym fallbackiem na port 3002.
        \end{itemize}
    \item Middleware
        \begin{itemize}
            \item CORS: Konfiguracja Cross-Origin Resource Sharing (CORS) ogranicza żądania do zaufanych domen (tutaj http://localhost:5173), zezwalając na metody HTTP takie jak GET, POST, DELETE itp. Jest to istotne z punktu widzenia 
            \item JSON Parser: Użycie express.json() jako middleware do parsowania przychodzących żądań z treścią w formacie JSON.
        \end{itemize}
    \item Routing
        \begin{itemize}
            \item Read Routes: Import i użycie route'ów do odczytu danych (readRoutes) pod ścieżką /api/read.
            \item Create Routes: Import i użycie route'ów do tworzenia danych (createRoutes) pod ścieżką /api/create.
            \item Delete Routes: Import i użycie route'ów do usuwania danych (deleteRoutes) pod ścieżką /api/delete.
            \item Update Routes: Import i użycie route'ów do aktualizacji danych (updateRoutes) pod ścieżką /api/update.
        \end{itemize}
    \item Generowanie Tokenów JWT
        \begin{itemize}
            \item Tokeny te są wykorzystywane do uwierzytelniania i zarządzania sesjami użytkowników.
        \end{itemize}
    \item Konfiguracja ORM Prisma
        \begin{itemize}
            \item Prisma jest ORM (Object Relational Mapping) do Node.js i TypeScript, który umożliwia bezpośredni dostęp do bazy danych z poziomu kodu źródłowego aplikacji.
            \item Prisma Client jest generowany na podstawie schematu bazy danych, który jest zdefiniowany w pliku schema.prisma.
            \item Użyto MySQL jako bazy danych.
            \item Prisma jest skonfigurowana do połączenia z bazą danych za pomocą zmiennej środowiskowej DATABASE\_URL.
        \end{itemize}
\end{enumerate}