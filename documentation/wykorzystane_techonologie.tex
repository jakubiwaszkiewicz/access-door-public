\section{Wykorzystane technologie}

\noindent Wykorzystana technologia: System kontroli wersji - Git

System kontroli wersji, który został wybrany do projektu to git. 
W znaczym stopniu ułatwił on pracę nad projektem. 
Pozwolił na łatwe śledzenie zmian w kodzie, 
a także na łatwe przywracanie poprzednich wersji. 
Dzięki niemu możliwa była również praca nad projektem 
przez dwie osoby jednocześnie. 

Stworzyliśmy repozytorium na platformie github.com.

\subsection{Wykorzystane technologie programistyczne}

\begin{itemize}
    \item \textbf{React.js}: Biblioteka React.js, która została wykorzystana do stworzenia i ustrukturyzowania interfejsu użytkownika przy pomocy JSX'a.
    \item \textbf{Express.js}: Biblioteka express.js, która została wykorzystana do stworzenia serwera, API i endpointów.
    \item \textbf{SASS}: Biblioteka SASS, która została wykorzystana do stylizacji interfejsu użytkownika.
    \item \textbf{Prisma}: ORM, który został wykorzystany do połączenia się, wykonywania zmian jak i komunikacji z bazą danych.
    \item \textbf{AWS}: Chmura AWS, która została wykorzystana do hostowania bazy danych.
    \item \textbf{REST}: Architektura REST, która została wykorzystana do komunikacji między serwerem a klientem. 
    \item \textbf{Figma}: Program Figma, który został wykorzystany do stworzenia projektu interfejsu użytkownika.
    \item \textbf{Visual Studio Code}: Środowisko programistyczne Visual Studio Code, które zostało wykorzystane do pisania kodu serwera, kodu aplikacji jak i kodu dokumentacji.
    \item \textbf{LaTeX}: Środowisko programistyczne LaTeX, które zostało wykorzystane do pisania dokumentacji.
\end{itemize}
