\section{Opis projektu}
\subsection{Skład grupy}
\begin{enumerate}
    \item Jakub Iwaszkiewicz - React.js, Node.js, MySQL, Express.js
    \item Krzysztof Marszałek - React.js, styles, MySQL
\end{enumerate}

\subsection{Cel projektu}
\noindent Celem projektu jest stworzenie nowoczesnej 
aplikacji internetowej, która poprawi efektywność i wydajność 
zarządzania dostępem do drzwi. Aplikacja "dostęp do drzwi" jest dla zakładu 
pracy i umożliwia dostęp do drzwi dla pracowników, który obecnie są w nim zatrudnieni. 
Aplikacja jest przeznaczona dla intranetu danego zakładu pracy. Użytkownicy niezalogowani 
mogą zalogować się jako pracownicy lub jak administrator, w zależności od tego czy dany 
użytkownik odpowiada za dostęp do drzwi. Administrator ma dostęp do wszystkich informacji 
jak i usług w firmie, czyli: może przyznawać dostępy do drzwi dla pracowników, może także 
usuwać te dostępy. Dodatkowo w każdej chwili może zmienić uprawienenia dostępu do drzwi w 
zależności od różnych sytuacji. Administrator dostaje prośbe o dostęp do drzwi w formie 
powiadomienia w momencie kiedy pracownik ją wyśle. Może zaakceptować albo anulować request. 
Co również odróznia administratora to fakt, że ma on wgląd w całą historie przyznawania 
dostępu do drzwi, przez siebie jak i przez innych administratorów. Pracownik może odcztać
 jakie ma przyznane dostępy do drzwi oraz wysłać prośbe o dostęp do nowych i tę prośbe 
 również usunąć. Projekt ma na celu nie tylko stworzenie funkcjonalnej 
 aplikacji, ale także zdobycie doświadczenia w pracy zespołowej i rozwijanie 
 umiejętności w zakresie nowoczesnych technologii webowych.

 \subsection{Typy użytkowników}
    \begin{enumerate}
        \item \textbf{Właściciel} \\
Właściciel jest użytkownikiem, który ma dostęp do wszystkich funkcjonalności systemu. Może on zarządzać użytkownikami, drzwiami, dostępami do drzwi oraz administratorami.


        \item \textbf{Administrator}\\
Administrator jest użytkownikiem, który ma dostęp do wszystkich funkcjonalności systemu. Może on zarządzać użytkownikami, drzwiami oraz dostępami do drzwi.       

        \item \textbf{Pracownik} \\
Pracownik jest użytkownikiem, który ma dostęp do uprawnień przyznanych przez administratora. 
        \item \textbf{Niezalogowany użytkownik} \\
Niezalogowany użytkownik nie ma dostępu do żadnych funkcjonalności systemu. Może on jedynie zalogować się do systemu.
    \end{enumerate}