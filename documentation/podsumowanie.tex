\newpage
\section{Podsumowanie}

\subsection{Wnioski}

Tworzenie aplikacji internetowej czy aplikacji mobilnej to 
bardzo złożona procedura wymagająca od progamisty bardzo złożonej 
wiedzy na temat narzędzi jak i samego wykorzystania narzędzia. 
Naszym ułatwieniem było to że jedna osoba z grupy już wcześniej 
przez jakiś czas miała do czynienia z narzędziem React.js co 
pozwoliło nam na w miare swobodne nauczenie się Express.js'a 
który pozwolił nam na ułożenie sobie wszystkiego w głowie a 
potem przelanie wszystkiego na kod. 
Pierwszą rzeczą jaką musieliśmy zrobić to wygląd graficzny 
naszej aplikacji która została wykonana w Figmie - 
samo znalezienie odpowiedniego narzędzia do tego i 
nauczenie się go było trudnym zadaniem, a 
co dopiero znalezienie palety kolorów, 
czcionek czy stylu naszej aplikacji - 
która i tak w trakcie programownia 
została nieco zaniechana przez brak czasu. 
Kolejnym wyzwaniem było rozpisanie funkcjonalności
jak i zostosowanie się do nich i wykorzystanie ich w Expressjs'ie.
Następnym wyzwaniem było połączenie się z bazą danych za pomocą Prisma,
co znacznie ułatwiło nam pracę z backednem.
W naszym projekcie React.js odegrał fundamentalną rolę, stanowiąc kręgosłup interfejsu użytkownika. Wykorzystanie tej biblioteki umożliwiło nam stworzenie dynamicznych, interaktywnych i responsywnych widoków, które dopasowują się do różnorodnych potrzeb użytkowników. Dzięki modularnej naturze React, nasz zespół był w stanie efektywnie zarządzać kodem, tworząc reużywalne komponenty, które znacząco przyspieszyły rozwój aplikacji.

Kluczowym elementem naszej pracy z React.js było zastosowanie zaawansowanych funkcji zarządzania stanem i cyklu życia komponentów. Hooki takie jak useState i useEffect odegrały zasadniczą rolę w dynamicznym zarządzaniu danymi aplikacji, umożliwiając natychmiastową reakcję interfejsu na zmiany.

Następnym krokiem było stworzenie podstron dla każdego rodzaju użytkownika 
oraz zaimpelementowanie React Router Dom, który pozwolił nam na
przekierowanie użytkownika na odpowiednią podstronę w zależności od jego uprawnień.
Kolejnym krokiem było stworzenie interfejsu graficznego dla każdej podstrony.
Wykorzystaliśmy do tego bibliotekę SASS, która pozwoliła nam na
zastosowanie styli w naszej aplikacji. 


\subsection{Możliwości rozwoju}

Dzięki zastosowaniu express.js'a jako tak na prawdę osobnej aplikacji, skalowalność aplikacji jest bardzo prosta, od strony serwerowej na pewno zabrakło możliwości napisania o odzyskanie zgubionej karty, jak i możliwości otworzenia drzwi w sposób zdalny (na stronie a nie przy pomocy karty) przez pracownika. Kolejną rzeczą jaką bym dołożył, to zwiększona ilość przechowywanych informacji przez naszą bazę danych pod względem wykonywanych działań przez pracowników (zapisywanie w bazie danych informacji o tym że ktoś przeszedł o tej godzinie przez te drzwi) jak i informacji na temat wykonywanych działań przez administratorów (zapisywanie informacji o tym że ktoś dodał pracownika lub właściciel został zmieniony również widoczny dla wszystki administratorów).

Jeżeli chodzi o interfejs graficzny to przez brak czasu niektóre elementy nie zostały w 100\% wystylizowane i niektóre mogą wyglądać nieco inaczej niż wszystkie - właśnie przez nie wykorzystanie pełnego potencjału komponentów reactowych i oddelegowywanie ich i wykorzystywanie we wszystkich podstronach, gdzie czasem mamy komponent w tym samym pliku co element wykorzystujący komponent, zamiast ulokowanego w innym pliku, gdzie jest to standardowa procedura dla Reacta.

Sama aplikacja może zostać rozwinięta przez jeszcze bardziej zaawansowane elementy, takie jak odzyskiwanie hasła, dwuetapowa weryfikacja, wysyłanie wiadomości do współpracowników, lecz zabrakło nam niestety czasu. Jeżeli pozwoli nam motywacja to prawdopodobnie będziemy chcieli ją dostosować do standardów rekrutacyjnych i będzie naszą pomocą w znalezieniu praktyk na wakacje.
\newpage
\listoffigures
\listoftables